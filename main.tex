\documentclass[12pt]{article}
\usepackage[utf8]{inputenc}
\usepackage[T1]{fontenc}
\usepackage[french]{babel}
\usepackage{amsmath,amsfonts,amssymb}
\usepackage[utf8]{inputenc}
\usepackage{mathptmx}
\usepackage{graphicx}
\usepackage[left=3cm,right=3cm, top=4cm, bottom=4cm]{geometry}
\begin{document}
\begin{figure}
\includegraphics[width=.5\textwidth]{cerdi.jpg}
\includegraphics[width=.5\textwidth]{download.png}
\end{figure}
\title{Rapport de Biens Publics Mondiaux}
\author{OUEDRAOGO Adama }
\date{May 2020}

\maketitle
\begin{center}
\newline
\paragraph{}
\paragraph{}
\textit{Chargé du cous: M.Grégoire ROTA-Graziosi}
\end{center}
\newpage


\maketitle

\section{INTRODUCTION}
\par  La protection de l’environnement est présenté comme un bien public mondial qui appelle une coopération entre les différents États dans le monde. ces coopérations se traduisent par des accords internationaux sur l’environnement afin de déterminer comment financer ce bien public, les technologie d'agrégation et le mode fourniture .Ces accords sont dit auto-exécutoires parce qu'on ne peut pas être faire appel à une autorité supérieure pour l’exécution : les termes de l'accord doivent être tels que l'exécution et les incitations à adhérer à l'accord sont implicites.  Les accords internationaux en matière d'environnement sont importants, puisqu'ils permettent à différents pays de travailler ensemble pour trouver des solutions aux enjeux environnementaux cruciaux ayant un caractère mondial, notamment la pollution atmosphérique, les changements climatiques, la protection de la couche d'ozone et la pollution des océans. 
\par Par ailleurs, il est difficile de forger un consensus international sur la protection de l'environnement parce que non seulement, tous les signataires doivent volontairement adhérer au traité mais aussi il existe des différences substantielles d'un pays à l'autre en termes de coûts et d'avantages. A l’équation de comment convenir un accord international pour le bien-être de tous s’ajoute aussi celle de répondre à la question du « passager clandestin » : comment empêcher un payer de bénéficier d’un accord sans adhérer ? Il ressort pertinent de comprendre ce qu'il faut pour soutenir un accord de ce type qui améliore le bien-être mondial et auquel un nombre important de pays chercherons volontairement à participer à cet accord. Plusieurs facteurs influencent la décision d’un pays d’appartenir à un accord ou non.  Parmi ces facteurs, on note l’incertitude et l’apprentissage. Alors, comment ces facteurs affectent-ils la formation d’un accords environnemental international auto-exécutoire ? il est important également d’évoquer la question de la stabilité d’un accord en se référant à la taille et à ce qui maintien les pays ans de l’accord. C'est l'objet de présent rapport.
\par Nous nous basons sur les travaux antérieurs afin d'étudier les incertitudes systématiques dans les accords environnementaux internationaux auto-exécutoires. Notre article des base en l'occurrence celui de Charles D. KOLSTAD propose un modèle standard d'accords environnementaux auto-exécutoire auquel il introduit l'incertitude et l'apprentissage.  Cette incertitude concernant les avantages et les coûts peut être résolue entre le moment où un pays s'engage à conclure un accord international et le moment où les pays signataires décident des niveaux d'émission. Nous constatons au cours de cette analyse que l'incertitude et l'apprentissage peuvent modifier la taille d'un accord international sur l'environnement. Dans la suite de notre travail, nous allons présenter la méthode utilisée par l'auteur (section 2) ainsi que les résultats (section 3). la quatrième section sera consacré à une analyse critique  et nous finirons par la conclusion.

\section{PRÉSENTATION DE LA MÉTHODOLOGIE}
\par L'auteur considère un modèle d'accords environnementaux internationaux standard inspiré d'Ulph (2004) comme modèle de base auquel il introduit successivement l'incertitude systématique (incertitude sur une variable qui est commune à tous les pays) et l'apprentissage, pour voir comment ces paramètres vont influencer la taille de l'accord ainsi que les décisions d'émissions des différents membres. En effet, on considère $i$ pays émettant chacun un niveau de pollution noté $q_{i}$. La quantité de pollution à émettre est un choix discret limité à $q_i=0$ pour une réduction et à $q_i=1$  lorsqu'il s'agit de la pollution.Dans ce cas le gain de chaque pays identique est représenté comme une fonction linéaire des émissions propres et des émissions agrégées:
$\Pi_i(q_i,Q_{-i})=cq_i-bQ=q_i-\gamma(q_i-Q_{-i})$
avec $( Q_{-i} = \Sigma_{i\neq j} q_j )$ et $\gamma=b/c$ et on suppose que $c=1$.
\newline $\gamma$ étant le rapport bénéfice-coûts de l'émission, il est considéré dans la résolution, le cas où $\gamma<1$.
\par La formation d'un accord international se présente alors comme un système de jeu en deux étapes à savoir un jeu d'adhésion à l'accord puis un jeu d'émission qui décrit les quantités à émettre. Dans ce dernier cas les membres décident ensemble des émissions et les non membres décident individuellement et de manière non coopérative de sorte qu'on débouche sur un équilibre de Nash. La stabilité interne (aucun pays n'a d'incitation à quitter l'accord) et la stabilité externe (aucun pays n'a d'incitation à rejoindre l'accord) sont exprimées en termes de gains auxquels les membres de la coalition et les non membres peuvent s'attendre.
\par Dans un second temps l'auteur introduit l'incertitude systématique sur $\gamma$ dans le modèle en considérant deux États de telle sorte que la valeur de $\gamma$ varie dans chacun des pays. Autrement le niveau d'incertitude crée une différence du rapport avantages-coûts et qu'une fois cette incertitude résolu, tous les pays seront donc identiques en terme d'avantages-coûts de l'accord.Trois niveaux d'apprentissage est à distinguer dans le cadre de cette analyse. Celui où les actions peuvent être entièrement conditionnées par les états du monde ou interprétés alternativement, que l'apprentissage se produit avant à la fois l'adhésion et le jeu des émissions (que nous appelons apprentissage complet), celui où les actions ne peut pas être conditionné aux états du monde et l'incertitude n'est jamais résolue (incertitude sans apprentissage), et l'incertitude est résolue entre les jeux d'adhésion et d'émissions (apprentissage partiel). La taille de la coalition sera modifiée selon que l'on soit dans une situation ou d'une autre et  en présence d'incertitude.

\section{RÉSULTATS D'ANALYSE DU BIEN PUBLIC}
\par L'analyse des résultats tiendra compte des facteurs incertitudes et apprentissages intégrées dans ce modèle. Ce qui nous permettra de savoir comment ces accords sont-ils formés, comment les pays prennent leurs décision en matière de fourniture du bien public qui est la préservation de l'environnement dans ce cas.
D'une part pour ce qui concerne le jeu d'annonce et d'émission, il existe un équilibre stable caractérisé par un nombre constat de pays dans l'accord tel que $n*=I(1/\gamma)$ avec $n$, le nombre de membre de la coalition. En effet, pour $n<1/\gamma$, on remarque que les gains issus des membres de l'accord et les gains des non membres sont identiques. Dans ce cas tout le monde décide alors de polluer et la coalition serait donc inefficace. Par ailleurs pour $n>1/\gamma$, l'accord est bénéfique pour les membres et ils peuvent facilement empêcher le "free riding" des non membres jusqu'à ce que le nombre de membre soit égal à l'inverse du rapport avantages-coûts et pays opte pour "tout le monde polluant" et leurs gains seront très bas. En l'absence d'incertitude, un accord de taille $n$ est stable intérieurement lorsque les gains des membres dépassent celui des non membres. En revanche un accord serait dit stable extérieurement si et seulement si les gains escomptés de l'accord suite à l'adhésion d'un nouveau membre reste inférieur aux gains des non membres.Étant donné que $\gamma$ est le rapport des dommages environnementaux aux coûts de réduction, cela signifie que si les dommages sont faibles (par rapport aux coûts de réduction), une grande coalition est susceptible de se former, bien que les dommages et les coûts de réduction ne justifieront pas beaucoup d'action.Par contre si les dégâts sont importants, la coalition sera probablement petite, avec peu d'action non plus. De toute façon, la coalition n'aide pas beaucoup dans ce cas.
\par En introduisant le concept d'incertitude et de l'apprentissage à trois niveau, l'auteur à aboutit à un certains nombre de résultat qui sont les suivants:
\newline D'abord, dans une situation d'apprentissage complet, les actions d'un membre peuvent être conditionnées par les autres états du monde. En d'autres termes, l'information tend à augmenter la taille de la coalition stable. Cela s'explique par le fait que les différents pays dans l'accord porte déjà un grand espoir à l'accord et se donne une idée des l'avenir qui promet des des avantages de d'accords pour les membre que les non membre.
\newline Ensuite dans une situation d'apprentissage partielle avec incertitude, nous constatons que les pays s'engagent à appartenir dans l'accord afin de contrôler la pollution, mais l'apprentissage se produit et la coalition décide du niveau de contrôle des émissions à entreprendre. Alors, l'apprentissage peut réduire la taille et la stabilité de l'accord mais un autre accord existe toujours: c'est le cas où le rapport coûts-avantages serait donc très élevé. Autrement dit l'apprentissage a tendance à entraîner l'émergence d'une coalition stable plus grande de participants à un accord environnemental international. Par ailleurs, lorsqu'un besoin de coalition ou d'émission se fait sentir, il est est probable que l'apprentissage partielle abouti à une plus petite taille de l'accord.
\newline Enfin en situation de non apprentissage,le problème de l'incertitude ne saurait être résolu car cette situation correspond au moment ou tout les décisions d'émission et de réduction sont déjà prises. En effet, on se retrouve donc dans une situation où les actions des pays ne peuvent pas être contrôlées vis à vis des autres. Mais  néanmoins dans cette situation il y a des décisions d'émissions et réduction de la pollution.
\section{ANALYSE CRITIQUE}
Dans cette section nous nous basons sur l'article de Thomas Eichner et al. "Self-enforcing environmental agreements and international trade" pour fonder une analyse critique de ce présent article. En le moèle utilisé est un modèle statique standard d'accord environnementaux internationaux, donc on pourrait s'attendre à ce que la variable "apprentissage" ne puisse pas décrire l'effet souhaité dans ce cas car l'apprentissage en général est un processus dynamique. De plus les accords environnementaux sont régis par des mécanisme d'incertitude aussi substantielles qui n'ont pas été ressorti dans ses analyse. Aussi, l'hypothèse d'émissions non essentielles n'est pas pleinement satisfaisante pour les émissions de carbone dans le contexte de l'atténuation du changement climatique. 
\par Par contre dans le deuxième article,tout en intégrant la variable commerce international dans le mécanisme des accords environnementaux internationaux, l'auteur étend son modèle en introduisant un bien de consommation composite et des combustibles fossiles qui sont produits et consommés dans chaque pays et négociés sur les marchés mondiaux. Ainsi de grands accords peuvent se former favorisant donc  coopération réussie et efficace dans la lutte contre le changement climatique. Dans son modèle il pourra mieux capter les phénomène d'émission et de négociation à travers le poids de ces variables dans chaque pays.
\newline Enfin dans les questions de la protection de l'environnement comme bien public mondial, la fourniture se fait à travers le concept de maillons faibles. C'est-à-dire que pour les accords, les plus faibles auront tendance beaucoup négocier car c'est eux qui sont les plus fragile face à ce problème.

\section{CONCLUSION}
\par Les accords environnementaux internationaux sont importants car ils permettent aux différents acteur de trouver des solutions à la question de l'environnement. Les différences substantielles des pays rendent difficile un consensus international. Ce document nous permis de comprendre dans quelle mesure la taille d'un tel accord pourrait changer sous l'influence des variables telles que l'incertitude systémique et l'apprentissage. En effet en situation d'apprentissage complet, la valeur de l'information aura augmentera la taille de l'accord. En situation d'apprentissage partielle, la taille de l'accord aura tendance à être réduit. Cependant, en situation de non apprentissage, il y a des décisions qui sont prises mais le problème de l'incertitude n'est pas résolu. Au regard des différents limites du modèle, de nombreux prolongements de ce travail pourrait être envisagé, en explorant cette thématique sous d'autres angles.
\end{document}
